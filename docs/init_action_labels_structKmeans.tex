\documentclass[11pt,letterpaper]{article}

\usepackage{times}
\usepackage{epsfig}
\usepackage{graphicx}
\usepackage{amsmath}
\usepackage{amssymb}
\usepackage{fullpage}
\usepackage{algorithm}
\usepackage{algorithmic}
\usepackage{verbatim}

% Include other packages here, before hyperref.

% If you comment hyperref and then uncomment it, you should delete
% egpaper.aux before re-running latex.  (Or just hit 'q' on the first latex
% run, let it finish, and you should be clear).
\usepackage[pagebackref=true,breaklinks=true,letterpaper=true,colorlinks,bookmarks=false]{hyperref}

\DeclareRobustCommand\onedot{\futurelet\@let@token\@onedot}
\def\@onedot{\ifx\@let@token.\else.\null\fi\xspace}
\def\eg{\emph{e.g}\onedot} \def\Eg{\emph{E.g}\onedot}
\def\ie{\emph{i.e}\onedot} \def\Ie{\emph{I.e}\onedot}
\def\cf{\emph{c.f}\onedot} \def\Cf{\emph{C.f}\onedot}
\def\etc{\emph{etc}\onedot} \def\vs{\emph{vs}\onedot}
\def\wrt{w.r.t\onedot} \def\dof{d.o.f\onedot}
\def\etal{\emph{et al}\onedot}
\DeclareMathOperator*{\argmin}{arg\,min}
\DeclareMathOperator*{\argmax}{arg\,max}

\newcommand{\+}[1]{\ensuremath{{\boldsymbol #1}}}


% Pages are numbered in submission mode, and unnumbered in camera-ready
\begin{document}

%%%%%%%%% TITLE
\title{Algorithm to initialize action labels for region-based actions using frame-level action annotations}

%\author{Juan Carlos Niebles\\
%Universidad del Norte\\
%Institution1 address\\
%{\tt\small njuan@uninorte.edu.co}
% For a paper whose authors are all at the same institution,
% omit the following lines up until the closing ``}''.
% Additional authors and addresses can be added with ``\and'',
% just like the second author.
% To save space, use either the email address or home page, not both
%}

\maketitle
%\thispagestyle{empty}

\section{Problem}

In $M$ videos, we have $Q_m$ temporal annotations of actions (action stamps) in each video, in the form of a time interval where the action is performed. The actions can overlap in time. We wish to split each frame into $R$ regions, and select a single action label for each region in each frame. As we do not know in advance which regions perform the actions, we plan to build a model to associate each action interval with one or more regions.

The structure of the problem allow us to infer the following relations:
\begin{itemize}
\item Each action interval $q$ must appear at least one time in the video. The action label of the action interval $q$ is defined as $a_q$.
\item Two overlapping action intervals can't appear in the same region.
\item The maximum number of action intervals belonging to each region must be at last the number of non-overlapping actions in each video. This number is defined as $t_m$
\item The maximum number of regions sharing the same action interval $q$ must be the difference between the total number of regions ($R$) and the number of time overlapping actions with $q$. This number is defined as $o_q^m$
\end{itemize}
We define $v_{r,q}^m=1$ when the action interval $q$ appears in region $r$ in the video $m$, and $v_{r,q}^m=0$ otherwise. We assume we have pose labels for each frame, independent for each region. For an action interval $q$, we use the histogram of pose labels for each region in the action interval, defined for the video $m$ as $h_{r,q}^m$ . We can solve the problem of finding the correspondence between action intervals and regions in a formulation similar to $k$-means, using the structure of the problem as constraints in the labels. 
\begin{equation}
\begin{split}
P1) \quad \min J= &\sum_{m=1}^M  \sum_{r=1}^R \sum_{q=1}^{Q_m}  v_{r,q}^m || h_{r,q}^m - \mu_{a_q}^r||_2^2 -\frac{1}{\lambda} v_{r,q}^m\\ 
 \text{s. to} 
\quad 
& 1 \le \sum_{r=1}^R v_{r,q}^m \le o_q^m \\ 
& \sum_{q=1}^{Q_m} v_{r,q}^m \le t_m \\ 
& v_{r,q_1}^m + v_{r,q_2}^m \le 1 \text{ if } q_1\cap q_2 \neq \emptyset \\  
& v_{r,q}^m \in \{0,1\},~\forall m
\end{split}
\end{equation}
 
$\mu_{a_q}^r$ are computed as the mean of the descriptors with the same action label within the same region, and $\mu_{idle}^r$ are computed using all the non-as. The general idea is to solve $P1$ iteratively as $k$-means,  finding the cluster centers for each region $r$, $\mu_{a}^r$ using the labels $v_{r,q}^m$, and then finding the best labeling given the cluster centers.

\section{Solving $P1$}

\subsection{Finding the optimal labels given $\mu_{a}^r$}

The most difficult step is to find the labels $v_{r,q}^m$ that minimizes $P1$ satisfying the constraints. 

\paragraph{Exhaustive search} One alternative is to make an exhaustive search over the set of valid labeling. For instance, if there is $Q$ non-overlapping actions (the worst case), then the number of different labeling matrices is $16^Q$, which can be handled if $Q$ is small. When action intervals overlap, the possibilities decrease and possibly the problem can still be handled using exhaustive search. If there is small overlapping of action intervals, and Q is moderate, then alternative methods should be used.

\paragraph{Relaxing the problem} Other alternative is relaxing the labels to lie in the $[0,1]$ interval, transforming the integer program into a linear program, that can be solved in polynomial time. Also, a sparsity regularization  term added to the relaxed variables could be useful to search for better solutions, keeping the problem as a linear program formulation. There is some alternatives to refine the solutions if needed:

\begin{itemize}
\item Branch and bound techniques for fractional values
\item Cutting plane method for integer programming
\end{itemize}

\subsection{Finding the cluster centers}
Given the labels, finding the cluster centers seems to be straightforward. Nevertheless, as some frames ``turned off'' if their  label become 0, there is still some doubts about the convergence rate of this algorithm, that should be explored with caution. 

\section{Using classifiers scores instead of distortion metric}
We could change the objective function to learn classifiers instead of cluster centers, maximizing the score of the classifiers instead of minimizing the distortion. One disadvantage is that J is unbounded, and we are maximizing, so we would need tor regularize $W$. For instance, a related formulation could be given by

\begin{equation}
\begin{split}
P2)\quad \max J= &\sum_{m=1}^M \sum_{r=1}^R \sum_{q=1}^{Q_m} v_{r,q}^m  {W_{a_q}^r}^\top h_{r,q}^m  - \lambda \Omega(W)\\  \\  \text{s. to} \quad & 1 \le \sum_{r=1}^R v_{r,q}^m \le o_q^m \\ & \sum_{q=1}^{Q_m} v_{r,q}^m \le t_m \\ & v_{r,q_1}^m + v_{r,q_2}^m \le 1 \text{ if } q_1\cap q_2 \neq \emptyset \\  & v_{r,q}^m \in \{0,1\},~\forall m
\end{split}
\end{equation}


\end{document}
