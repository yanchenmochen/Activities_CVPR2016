We present a novel hierarchical compositional model to recognize
human activities using RGB-D data. The proposed method
is able to jointly learn suitable representations at different abstraction
levels leading to compact and robust models, as shown by the experimental
results. In particular, our model 
achieves multi-class discrimination while
providing useful annotations at the intermediate semantic level.
The compositional capabilities of our model also bring robustness to partial
body occlusions.

Through our experiments, we show how a fused descriptor, composed of geometric
features and motion descriptors, improves the accuracy of the
activity prediction by over 5\% compared to the geometric descriptor alone in
the Composable Activities dataset. Moreover, the model makes an efficient use of
learned body poses by imposing sparsity over coefficients of the mid-level
(atomic action) representations. We observe that this capability produces
specialization of poses at the higher levels of the hierarchy. Recognition
accuracy using sparsity over mid-level classifiers is similar to the case of
using a quadratic regularizer alone, however, the model semantic interpretation
is improved, since the interactions of the pose classifiers with the upper
levels are more efficient, measured throughout the influence of poses over the
activities.

There are several research avenues for future work. In particular, during
training our current model requires annotated data at the level of action,
which can be problematic for a large scale application. An improvement could be
treating the action labels as latent variables, and using only a list of
possible action labels for every activity, similarly to works of recognizing
objects by text tags. Also, for real-time video recognition, we also need to
include inference with respect to the temporal position and span of each
activity, which can be also considered as latent variables. Finally, as we
mentioned before, our model can be extended to the case of identifying the
composition of novel activities that are not present in the training set. 

