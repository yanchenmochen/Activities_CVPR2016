There is a large amount of work on human activity recognition in the computer
vision community \todo{Cite surveys}. 
In particular, we focus on the problem of recognizing human actions and
activities from videos using pose-based representations.

\paragraph{Problem} Pose-based action recognition from color videos, and
from depth videos.
Some tackle the problem of jointly recognizing actions and poses in videos
\cite{Nie2015} and in still images \cite{Yao2010}.
Skeletons in a Lie Group \cite{Vemulapalli2014}

\paragraph{Scope} From simple actions to concurrent and composable activities.
Dense annotation of actions \cite{Yeung2015}.
Fine-grained cooking dataset \cite{Rohrbach2012}.

\paragraph{Descriptors} Fusing poses with other descriptors. P-CNN \cite{Cheron2015}
fuses pose-centered CNN features extracted from optical flow and color.
Fusion of color and depth features: Heterogeneous features \cite{Hu2015};
Bilinear heterogeneous information machine \cite{Kong2015};

\paragraph{Representation} Poselets.
Moving Poselets \cite{Tao2015}.
Dynamic Poselets \cite{Wang2014}.
Moving Pose, a descriptor for action recognition \cite{Zanfir2013}

\paragraph{Models} Latent variable models for action recognition.
Latent models for activity recognition \cite{Hu2014}.

\paragraph{Models} Neural Networks. RNN \cite{YongDu2015}










