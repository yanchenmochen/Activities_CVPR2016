\subsection{Learning} \label{subsec:learning}

\paragraph{Initial actionlet labels }A relevant step of our training process is to obtain a suitable initialization 
of latent variables. This is challenging task because, at each time 
interval, each labeled action can be associated with any of the possible 
subsets of R body regions. Fortunately, the machinery of self-paced 
learning \cite{Kumar:EtAl:2010} provides us with a suitable solution. Specifically, we 
formulate the problem of action to region 
assignment as an optimization problem, constrained using structural 
information: 
the actions intervals must not overlap in the same region, and all the action 
intervals must be present at least in one region. We formulate this labeling 
problem as a binary Integer Linear Programming (ILP) problem. We define as 
$v_{r,q}^m=1$ when the action interval $q \in \{1,\dots,Q_m\}$ appears in 
region 
$r$ of video $m$, and $v_{r,q}^m=0$ otherwise. We assume we have initial motion poselet 
labels 
$z_{t,r}$ in each frame, independent for each region. For an action interval $q$ and region $r$, we use as 
descriptor the histogram $h_{r,q}^m$ of motion poselet labels. We can solve the problem of finding 
the correspondence between action intervals and regions in a formulation 
similar 
to $k$-means using the structure of the problem as constraints in the labels:
\begin{equation}
\begin{split}
P1) \quad \min_{v,\mu} &\sum_{m=1}^M  \sum_{r=1}^R \sum_{q=1}^{Q_m}  v_{r,q}^m 
d( h_{r,q}^m - \mu_{a_q}^r) -\frac{1}{\lambda} v_{r,q}^m\\ 
 \text{s. to} 
\quad 
& \sum_{r=1}^R v_{r,q}^m \ge 1\text{, }\forall q\text{, }\forall m \\ 
%& \sum_{q=1}^{Q_m} v_{r,q}^m \le t_m \\ 
& v_{r,q_1}^m + v_{r,q_2}^m \le 1 \text{ if } q_1\cap q_2 \neq \emptyset 
\text{, 
}\forall r\text{, }\forall m\\  
& v_{r,q}^m \in \{0,1\}\text{, }\forall q\text{, }\forall{r}\text{, }\forall m
\end{split}
\end{equation}
with
\begin{equation}
d( h_{r,q}^m - \mu_{a_q}^r) = \sum_{k=1}^K (h_{r,q}^m[k] - 
\mu_{a_q}^r[k])^2/(h_{r,q}^m[k] +\mu_{a_q}^r[k]).
\end{equation}

$\mu_{a_q}^r$ are computed as the mean of the descriptors with the same action 
label within the same region. We solve $P1$ iteratively in a blosk descending acheme, alternating between solving $v_{r,q}^m$ with $\mu_{a}^r$ fixed, which have a trivial solution, and then fixing $\mu_{a}^r$ to solve $v_{r,q}^m$, relaxing $P1$ to solve a linear program. Note that th second term of the objective function of $P1$ resembles the objective 
function of \emph{self-paced} learning as in \cite{Kumar:EtAl:2010}, fostering to 
balance between assigning a single region to every action, towards assigning 
all 
possible regions to the action intervals when possible.  

\paragraph{Learning the model parameters}
To learn the model parameters, we state the problem as an Latent Structural SVM problem \cite{Yu:Joachims:2010}, with two sets of latent variables for motion poses $Z$ and actionlets $V$. The energy funcion act as constraints of the model, to bias the learning of correct labeling of motion poselets and actionlets. 

We aim to find
optimal values for the energy parameters, as well as,
slack variables $\xi_i$ motion poselet labels $Z_i$ and actionlet labels $V_i$, by solving the following learning problem:
\begin{equation}
\label{eq:big_problem}
\min_{W,\xi_i,~i=\{1,\dots,M\}}    \frac{1}{2}||W||_2^2 + \frac{C}{M} \sum_{i=1}^M\xi_i ,
\end{equation}
where
\begin{equation}
W^\top=[\alpha^\top, \beta^\top, w^\top, \gamma^\top, \eta^\top, \theta^\top],
\end{equation}
and
\begin{equation} \label{eq:slags}
\begin{split}
\xi_i = \max_{Z,V,y}  \{  & E(X_i, Z, V, y) + \Delta( (y_i,V_i), (y, V)) \\
 & - \max_{Z_i}{ E(X_i, Z_i, V_i, y_i)} \}, \; \;\; i\in[1,...M].	
\end{split}
\end{equation}
In Equation (\ref{eq:slags}), each slack variable
$\xi_i$ quantifies the error of the inferred labeling for the corresponding video $D_i$. We solve Equation (\ref{eq:big_problem}) iteratively using CCCP algorithm, solving for latent labels $Z_i$ and $V_i$ given model parameters $W$, temporal actionlet annotations and the complex actions of the videos, using dynamic programming (see Section \ref{subsec:inference}), and then solving for $W$ via 1-slack formulation and using Cutting Plane algorithm \cite{Joachims2009}. 

The loss function $\Delta((y_i,V_i),(y,V))$ punishes the inference errors in the training set during training. As the spatial ordering of actionlets is unknown (hence the latent actionlet formulation), but the temporal composition is know, we can compute a list $A_t$ of possible actionlets for frame $t$, and include  that information into the loss function as
\begin{equation}
\Delta((y_i,V_i),(y,V)) = \lambda_y(y_i \ne y) + \lambda_v\frac{1}{T}\sum_{t=1}^T 
\delta(v_t \notin A_t)
\end{equation}

