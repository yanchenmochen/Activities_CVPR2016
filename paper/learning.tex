\subsection{Learning} \label{subsec:learning}
A relevant step of our training process is to obtain a suitable inicialization 
of latent variables. This is challenging task because, at each time 
interval, each labeled action can be associated with any of the possible 
subsets of R body regions. Fortunately, the machinery of self-paced 
learning \cite{} provides us with a suitable solution. Specifically, we 
formulate the problem of action to region 
assignment as an optimization problem, constrained using structural 
information: 
the actions intervals must not overlap in the same region, and all the action 
intervals must be present at least in one region. We formulate this labeling 
problem as a binary Integer Linear Programming (ILP) problem. We define as 
$v_{r,q}^m=1$ when the action interval $q \in \{1,\dots,Q_m\}$ appears in 
region 
$r$ in the video $m$, and $v_{r,q}^m=0$ otherwise. We assume we have pose 
labels 
$z_{t,r}$ in each frame, independent for each region, learned via clustering 
the 
poses for all frames in all videos. For an action interval $q$, we use as 
descriptor the histogram of pose labels for each region in the action interval, 
defined for the video $m$ as $h_{r,q}^m$ . We can solve the problem of finding 
the correspondence between action intervals and regions in a formulation 
similar 
to $k$-means, using the structure of the problem as constraints in the labels, 
and using $\chi^2$ distance between the action interval descriptors and the 
cluster centers: 
\begin{equation}
\begin{split}
P1) \quad \min_{v,\mu} &\sum_{m=1}^M  \sum_{r=1}^R \sum_{q=1}^{Q_m}  v_{r,q}^m 
d( h_{r,q}^m - \mu_{a_q}^r) -\frac{1}{\lambda} v_{r,q}^m\\ 
 \text{s. to} 
\quad 
& \sum_{r=1}^R v_{r,q}^m \ge 1\text{, }\forall q\text{, }\forall m \\ 
%& \sum_{q=1}^{Q_m} v_{r,q}^m \le t_m \\ 
& v_{r,q_1}^m + v_{r,q_2}^m \le 1 \text{ if } q_1\cap q_2 \neq \emptyset 
\text{, 
}\forall r\text{, }\forall m\\  
& v_{r,q}^m \in \{0,1\}\text{, }\forall q\text{, }\forall{r}\text{, }\forall m
\end{split}
\end{equation}
with
\begin{equation}
d( h_{r,q}^m - \mu_{a_q}^r) = \sum_{k=1}^K (h_{r,q}^m[k] - 
\mu_{a_q}^r[k])^2/(h_{r,q}^m[k] +\mu_{a_q}^r[k]).
\end{equation}

$\mu_{a_q}^r$ are computed as the mean of the descriptors with the same action 
label within the same region. We solve $P1$ iteratively as $k$-means,  finding 
the cluster centers for each region $r$, $\mu_{a}^r$ using the labels 
$v_{r,q}^m$, and then finding the best labeling given the cluster centers, 
solving an ILP problem. Note that the first term of the objective function is 
similar to a $k$-means model, while the second term resembles the objective 
function of \emph{self-paced} learning as in \cite{Kumar2010}, fostering to 
balance between assigning a single region to every action, towards assigning 
all 
possible regions to the action intervals when possible.  

[IL: INCLUDE  FIGURE TO SHOW P1 GRAPHICALLY]
The loss function $\Delta((y_i,V_i),(y,V))$ can no longer depend in the value 
of $V_i$, since it is now a latent variable. But, as we do know the time span 
of the actions in all videos, we can compute a list $A_t$ of possible actions 
for frame $t$, and transform the original loss function into
\begin{equation}
\Delta(y_i,(y,V)) = \lambda_y(y_i \ne y) + \lambda_v\frac{1}{T}\sum_{t=1}^T 
\delta(v_t \notin A_t)
\end{equation}
and use the same group of actions for each frame to impute the labels of 
actions 
and poses for each frame during the inference of latent variables

The model parameters are learned with a Latent Structural SVM formulation, 
iterating between searching the best assignments of poses $Z$ given the model 
parameters, and finding the best classifiers given the poses $Z$:
\begin{equation}
 Z_i^* = \max_{Z_i} E(X_i, Z_i, V_i, y_i)
\end{equation}