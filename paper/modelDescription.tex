\begin{figure}[tb]
\begin{center}
%\fbox{\rule{0pt}{2in} \rule{0.9\linewidth}{0pt}}
\includegraphics[width=0.99\linewidth]{./Fig/modelo.pdf}
\end{center}
   \caption{Graphical representation of our discriminative hierarchical model for recognition of composable human activities.
At the top level, activities are represented as compositions of atomic actions that are inferred at
the intermediate level. These actions are in turn compositions of poses at the
lower level, where pose dictionaries are learned from data. Our model also learn
temporal transitions between consecutive poses and actions. Best viewed in
color.}
\label{fig:overview}

\end{figure}


In this section, we introduce our model for pose-based recognition of complex
human actions. Our goal is to give the model the capability of annotating input
videos with the actions being performed at test time. In particular, we are
interested in automatically identifying the parts of the body that are involved
in each action (spatial localization), as well as the temporal span of each
action (temporal localization). Since we are interested in concurrent and
composable activities, we would also like to encode multiple levels of
abstraction, so that we can encode poses, actions and their compositions.
Therefore, we develop a hierarchical compositional framework for modeling and
recognizing complex human actions.

One of the key contributions of our model is its capability of spatially
localizing the body regions that are involved with the execution of each
activity, \emph{both at training and testing time}. This is, our training
process does not require careful spatial annotation and localization of
actions in the training set. Instead, our model can automatically
discover these spatial body regions that are relevant to each action
from temporal annotations of actions alone. In the following, we introduce
the components of our model and the training process that achieves this goal.

\subsection{Hierarchical Activity Model for Weakly Supervised Discovery of
Relevant Moving Poselets}
TITLE IS A PLACEHOLDER

\paragraph{Training}

\paragraph{Inference}

\subsection{Capturing large action variabilities with a Mixture of Action
Classifiers}
TITLE IS A PLACEHOLDER

Multiple action classifiers

\subsection{Robustness to irrelevant, non-discrimiantive, or noisy poses}
TITLE IS A PLACEHOLDER

Garbage collector.


\todo[inline]{El texto que sigue es el original de ivan.}

Our model for complex action recognition follows a hierarchical compositional
approach based on pose and mid-level representations, such as \cite{Lillo2014}
or \cite{Tao2015} .  As a distinguishing idea from previous works, we include a
mechanism to infer not only the global activity label of the videos, but also
recover quality temporal and spatial annotations of actions and poses. To this
end, we explore three main ideas: first, we create method to automatically
infer mid-level annotations for localize actions spatially. Then, going towards
general action recognition, we describe a method to handle multiple classifiers
for the same action, fostering to create action models with better
representation power in hierarchical setups, where usually low and mid-level
components are shared throughout high level activities. Finally, to complement
the generalization of the model we aim to build discriminative low level
classifiers. When using pose dictionaries, there is a high chance that noisy
and non-informative poses are present when training low-level (pose)
classifiers. In models like  \cite{Tao2015}, where  video descriptors use
aggregated data of the complete video, the effect of non-informative poses is
mitigated compared to a frame-based model, even when the latter provide richer
interpretations. We build our model in a general hierarchical activity model
based on BoW representations, although the approach can be extended to
hierarchical models in general.

\subsection{Improved hierarchical model}
 %
[CVPR2014 MODEL SUMMARY] We start by briefly depicting the model presented in \cite{Lillo2014}.
They model an activity using three hierarchical levels: poses, atomic actions, and activities.  The human body is splitted into four fixed spatial regions, corresponding to overlapping joints for arms and legs. A geometric feature is computed in each frame. The model considers energy potentials for poses as linear classifiers applied to frame features, coded by a latent label $z_t \in \{1,\dots,K\}$ that relates the region feature for frame $t$ to one of $K$ pose classifiers. In the mid-level of atomic actions, the model use as descriptor the BoW (histogram) representation of labels $Z$ for frames belonging to each action, in order to learn $A$ atomic action linear classifiers. The action label of a frame is denoted as $v_t \in \{1,\dots,A\}$. Finally, in the higher level of activities, the model use the BoW representation of actions in the video, to learn a multiclass linear classifier. Adding all energy terms, using for the moment $R=1$ regions for simplicity, and including temporal transition terms, the energy equation for video $D$ in \cite{Lillo2014} is 
%
\begin{equation}
\label{eq:energy2014}
\begin{split}
E(D) = \left[ {\alpha_y}^\top h(V) + \sum_{a=1}^A {\beta_a}^\top h_a(Z) + \sum_{t=1}^T {w_{z_t}}^\top x_t \right.\\\left.+ {\gamma}^\top h(V^{-1},V) + {\eta}^\top h(Z^{-1},Z) \vphantom{ \sum_{a=1}^A} \right]
\end{split}
\end{equation}
%
where $h(V)$ is the histogram of atomic action labels, $h_a(Z)$ the histogram of pose labels at those frames labeled with action $a$, and $h(V^{-1},V)$ and $h(Z^{-1},Z)$ represents histograms for action and pose transitions respectively. The main idea is that the energy score for the video using the correct activity must be higher than every other activity. As during inference the labels $Z$ and $V$ are unknown, they must be inferred along with the activity label. Thus, an interesting property of this model is the per-frame nature of the energy function. The authors in \cite{Lillo2014} test their model in their own Composable Activities dataset, which include rich action labels annotated independently for four human regions (arms and legs). To date, this is the only action dataset with this level of spatial annotations. Our work is oriented in making this model more flexible and general.

[NEW MODEL]
We propose three improvements to the model to make it more practical and flexible: firstly, to relax the need of action annotations for each human region, we propose to use latent variables representing the assignment of actions to human body regions; secondly, to overcome the multimodal representations of actions, we model each action as a group of linear classifiers instead of using a single classifier per action; and last, to handle the noisy or non-informative poses (NI), we propose a \emph{garbage collector} approach where the model itself identifies a threshold for the scores of pose classifiers of every human region. We describe each contribution in the following paragraphs.

\paragraph{Latent assignments of actions to human regions}

Including latent variables into the model formulation is relatively straightforward, although the main problem to solve is to get a proper initialization of the starting point for atomic actions, since there is a high chance to get sticked in a local minimum. We develop a method to get a proper initialization of action labels. We start defining action intervals, since we assume that the time span of each action is known. Then, we use structural information as non-overlap of actions in the same region, and constraining that at least one region must be assigned to each action interval. We also assume in this formulation that that poses for the same actions should be similar. We formulate the initial labeling as a binary Integer Linear Programming (ILP) problem. We define $v_{r,q}^m=1$ when the action interval $q$ appears in region $r$ in the video $m$, and $v_{r,q}^m=0$ otherwise. We assume we have pose labels $z_t$ for each frame, independent for each region. For an action interval $q$, we use the histogram of pose labels for each region in the action interval, defined for the video $m$ as $h_{r,q}^m$ . We can solve the problem of finding the correspondence between action intervals and regions in a formulation similar to $k$-means, using the structure of the problem as constraints in the labels. 
\begin{equation}
\begin{split}
P1) \quad \min J= &\sum_{m=1}^M  \sum_{r=1}^R \sum_{q=1}^{Q_m}  v_{r,q}^m || h_{r,q}^m - \mu_{a_q}^r||_2^2 -\frac{1}{\lambda} v_{r,q}^m\\ 
 \text{s. to} 
\quad 
& \sum_{r=1}^R v_{r,q}^m \ge 1 \\ 
%& \sum_{q=1}^{Q_m} v_{r,q}^m \le t_m \\ 
& v_{r,q_1}^m + v_{r,q_2}^m \le 1 \text{ if } q_1\cap q_2 \neq \emptyset \\  
& v_{r,q}^m \in \{0,1\},~\forall m
\end{split}
\end{equation}
$\mu_{a_q}^r$ are computed as the mean of the descriptors with the same action label within the same region. The general idea is to solve $P1$ iteratively as $k$-means,  finding the cluster centers for each region $r$, $\mu_{a}^r$ using the labels $v_{r,q}^m$, and then finding the best labeling given the cluster centers, solving an ILP problem. Note that the first term of $J$ is similat to a $k$-means model, while the second term resembles the objective function of \emph{self-paced} learning as in \cite{Kumar2010}, fostering to balance between assigning a single region to every action, towards assigning all possible regions to the action intervals when possible.  

[IL: INCLUDE  FIGURE TO SHOW P1 GRAPHICALLY]

\paragraph{Representing semantic actions with multiple atomic sequences}

As the poses and atomic actions in \cite{Lillo2014} model are shared, a single classifier is generally not enough to model multimodal representations, usual complex videos. We modify the original hierarchical model of \cite{Lillo2014} to include multiple linear classifiers per action. We create two new concepts: \textbf{semantic actions}, that refer to actions \emph{names} that compose an activity; and \textbf{atomic sequences}, that refers to the sequence of poses that conform an action. Several atomic sequences can be associated to a single semantic action, creating disjoint sets of atomic sequences, each set associated to a single semantic action.  The main idea is that the action annotations in the datasets are associated to semantic actions, whereas for each semantic action we learn several atomic sequence classifiers. With this formulation, we can handle the multimodal nature of semantic actions, covering the changes in motion, poses , or even changes in meaning of the action according to the context (e.g. the semantic action ``open'' can be associated to opening a can, opening a door, etc.). 

We first describe the semantic actions for each video using a normalized BoW representation of the initial assignments of poses $Z$. Then, we find a suitable number of atomic sequence classifiers for each semantic action. Inspired by \cite{Raptis2012}, we use the \emph{Cattell's Scree test} using the eigenvalues $\lambda_i$ of the affinity matrix of the semantic action descriptors, using $\chi^2$ distance. For each semantic action $u \in \{1,\dots,U\}$ we find the number of atomic sequences $G_u$ as $G_u = \argmin_i \lambda_{i+1}^2 / (\sum_{j=1}^i \lambda_j) + c\cdot i$, with $c=2\cdot 10^{-3}$. Then we cluster the BoW descriptors using k-means, using a different number of clusters for each semantic action $u$ according to $G_u$.

In the model, action and pose levels are almost the same as in \cite{Lillo2014} excepting that now we have a higher number of atomic actions. For the  activity level, the original model computes the histogram of actions as a video descriptor; in our model, the histogram is constructed in two steps, first aggregating the action sequence labels that belongs to the same semantic action, and then using the counts as the histogram for the activity classifiers $\alpha$. 

\paragraph{Towards a better representation of poses: adding a garbage collector}

The model in \cite{Lillo2014} uses all poses to feed action classifiers. Out intuition is that only a subset of poses in each video are really discriminative or informative for the actions performed, while there is plenty of poses that corresponds to noisy or non-informative ones. [EXPAND] Our intuition is that low-scored frames in terms of poses (i.e. a low value of $w_{z_t}^\top x_t$ in Eq. (\ref{eq:energy2014})) make the same contribution as high-scored poses in higher levels of the model, while degrading the pose classifiers at the same time since low-scored poses are likely to be related to non-informative frames. We propose to include a new pose, to explicitly handling those low-scored frames, keeping them apart for the pose classifiers $w$, but still adding a fixed score to the energy function to avoid normalization issues and to help in the specialization of pose classifiers. We call this change in the model a \emph{garbage collector} since it handles all low-scores frames and group them having a fixed energy score $\theta$. In practice, we use a special pose entry $K+1$ to identify the non-informative poses. The equation representing the energy for pose level is
%
\begin{equation} \label{Eq_poseEnergy}
E_{\text{poses}}  = \sum_{t=1}^T \left[  {w_{z_t}}^\top x_{t}\delta(z_{t} \le  K) + \theta 
\delta(z_{t}=K+1)\right] 
\end{equation}
where $\delta(\ell) = 1$ if $\ell$ is true and $\delta(\ell) = 0$ if
$\ell$ is false. The action level also change its energy:
\begin{equation}
\begin{split}
 \label{Eq_actionEnergy}
E_{\text{actions}} =  \sum_{t=1}^T \sum_{a=1}^A \sum_{k=1}^{K+1}  \beta_{a,k} \delta(z_t = k) \delta(v_t = a).
\end{split}
\end{equation}

Integrating all contribution detailed in previous sections, the model is written as:

[IL: EQUATIONS ARE NOT UPDATED YET]
Energy function:
\begin{equation}
E = E_{\text{activity}} + E_{\text{action}} + E_{\text{pose}}
  + E_{\text{action transition}} + E_{\text{pose transition}}.
\end{equation}

\begin{equation}
E_{\text{pose}} = \sum_{r,t} {w^r_{z_{t,r}}}^\top x_{t,r} = \sum_{r,t,k} {w^r_{k,r}}^\top x_{t,r} \delta_{z_{t,r}}^k
\end{equation}

\begin{equation}
E_{\text{action}} = \sum_{r,a} {\beta^r_{a}}^\top h^{a,r}(Z, V) = \sum_{r,a,t,k} \beta^r_{a,k} \delta_{z_{t,r}}^k \delta_{v_{t,r}}^a
\end{equation}

\begin{equation}
h_g^{r}(U) = \sum_{t} \delta_{u_{t,r}}^g
\end{equation}

So the energy in the activity level is
\begin{equation}
E_{\text{activity}} = \sum_{r} {\alpha^r_{y}}^\top h^{r}(U) = \sum_{r,g,t}  \alpha^r_{y,g} \delta_{u_{t,r}}^g
\end{equation}

\begin{equation}
E_{\text{action transition}} = \sum_{r,a,a'}  \gamma^r_{a',a} \sum_{t} \delta_{v_{t-1,r}}^{a'}\delta_{v_{t,r}}^a 
\end{equation}

\begin{equation}
E_{\text{pose transition}} =\sum_{r,k,k'}  \eta^r_{k',k}\sum_{t}\delta_{z_{t-1,r}}^{k'}\delta_{z_{t,r}}^{k}
\end{equation}





\subsection{Learning} \label{subsec:learning}

\textbf{Initial actionlet labels.} A relevant step of our training process is to obtain a suitable initialization 
of latent variables. This is challenging task because, at each time 
interval, each labeled action can be associated with any of the possible 
subsets of R body regions. Fortunately, the machinery of self-paced 
learning \cite{Kumar:EtAl:2010} provides us with a suitable solution. Specifically, we 
formulate the association problem between actions and body regions as an 
optimization problem. We constrain this optimization using two structural 
restrictions: i) actions intervals must not overlap in the same region, and 
ii) all action 
intervals must be present at least in one region. We formulate the labeling 
process as a binary Integer Linear Programming (ILP) problem. We define as 
$v_{r,q}^m=1$ when the action interval $q \in \{1,\dots,Q_m\}$ appears in 
region 
$r$ of video $m$, and $v_{r,q}^m=0$ otherwise. We assume that we have initial 
motion poselet 
labels 
$z_{t,r}$ in each frame, independent for each region. For an action interval $q$ and region $r$, we use as 
descriptor the histogram $h_{r,q}^m$ of motion poselet labels. We can solve the problem of finding 
the correspondence between action intervals and regions using a formulation 
that resembles the operation of the $k$-means algorithm, but using the 
structure of the problem as constraints in the labels:
{\small
\begin{equation}
\begin{split}
P1) \quad \min_{v,\mu} &\sum_{m=1}^M  \sum_{r=1}^R \sum_{q=1}^{Q_m}  v_{r,q}^m 
d( h_{r,q}^m - \mu_{a_q}^r) -\frac{1}{\lambda} v_{r,q}^m\\ 
 \text{s. to} 
\quad 
& \sum_{r=1}^R v_{r,q}^m \ge 1\text{, }\forall q\text{, }\forall m \\ 
%& \sum_{q=1}^{Q_m} v_{r,q}^m \le t_m \\ 
& v_{r,q_1}^m + v_{r,q_2}^m \le 1 \text{ if } q_1\cap q_2 \neq \emptyset 
\text{, 
}\forall r\text{, }\forall m\\  
& v_{r,q}^m \in \{0,1\}\text{, }\forall q\text{, }\forall{r}\text{, }\forall m
\end{split}
\end{equation}
with
\begin{equation}
d( h_{r,q}^m - \mu_{a_q}^r) = \sum_{k=1}^K (h_{r,q}^m[k] - 
\mu_{a_q}^r[k])^2/(h_{r,q}^m[k] +\mu_{a_q}^r[k]).
\end{equation}}

$\mu_{a_q}^r$ are computed as the mean of the descriptors with the same action 
label within the same region. We solve $P1$ iteratively using a block coordinate 
descending scheme, alternating between solving $v_{r,q}^m$ with $\mu_{a}^r$ 
fixed, which has a trivial solution, and then fixing $\mu_{a}^r$ to solve 
$v_{r,q}^m$, relaxing $P1$ to solve a linear program. Note that the second term 
of the objective function in $P1$ resembles the objective function of 
\emph{self-paced} learning \cite{Kumar:EtAl:2010}, managing the balance between 
assigning a single region to every action or assigning all possible regions to 
the respective action interval.  

\textbf{Learning model parameters.}
To learn the model parameters, we state the problem as an Latent Structural SVM 
problem \cite{Yu:Joachims:2010}, with two sets of latent variables for motion 
poselets $Z$ and actionlets $V$, respectively. The energy funcion constraints 
the model towards learning correct labels for motion poselets and 
actionlets. 

We find values for parameters in equations 
(\ref{eq:motionposelets}-\ref{eq:actionletstransition}), as well as,
slack variables $\xi_i$ motion poselet labels $Z_i$, and actionlet labels $V_i$, 
by solving the following learning problem:
{\small
\begin{equation}
\label{eq:big_problem}
\min_{W,\xi_i,~i=\{1,\dots,M\}}    \frac{1}{2}||W||_2^2 + \frac{C}{M} \sum_{i=1}^M\xi_i ,
\end{equation}}
where
{\small \begin{equation}
W^\top=[\alpha^\top, \beta^\top, w^\top, \gamma^\top, \eta^\top, \theta^\top],
\end{equation}}
and
{\small
\begin{equation} \label{eq:slags}
\begin{split}
\xi_i = \max_{Z,V,y}  \{  & E(X_i, Z, V, y) + \Delta( (y_i,V_i), (y, V)) \\
 & - \max_{Z_i}{ E(X_i, Z_i, V_i, y_i)} \}, \; \;\; i\in[1,...M].	
\end{split}
\end{equation}}
In Equation (\ref{eq:slags}), each slack variable
$\xi_i$ quantifies the error of the inferred labeling for the corresponding 
video $D_i$. We solve Equation (\ref{eq:big_problem}) iteratively using CCCP 
algorithm, solving for latent labels $Z_i$ and $V_i$ given model parameters $W$, 
temporal actionlet annotations, and the complex actions of the videos, using 
dynamic programming (see Section \ref{subsec:inference}), and then solving for 
$W$ via 1-slack formulation using Cutting Plane algorithm 
\cite{Joachims2009}. 

Loss function $\Delta((y_i,V_i),(y,V))$ penalizes inference errors during 
training. As the spatial ordering of actionlets is unknown (hence the latent 
actionlet formulation), but the temporal composition is know, we can compute a 
list $A_t$ of possible actionlets for frame $t$, and include  that information 
into the loss function as
{\small \begin{equation}
\Delta((y_i,V_i),(y,V)) = \lambda_y(y_i \ne y) + \lambda_v\frac{1}{T}\sum_{t=1}^T 
\delta(v_t \notin A_t)
\end{equation}}



\subsection{Inference}
The input to the inference algorithm is a new video sequence with features
$X$. The task is to infer the best activity label $\hat y$ and the best
atomic action labels $\hat V$. Additionally, we also need to estimate latent variables $Z$.
%{\small
\begin{equation}
  \hat y, \hat V, \hat Z = \argmax_{y, V, Z} E(X, Z, V, y)
\end{equation}
%}
We can solve this by the same equations for solving the most violated constraint during learning, setting %as in Equation (\ref{dp_recursion}), using 
$\lambda_i =0$, $i = \{1,2\}$, by exhaustively enumerating all values of activities $y$, and solving for atomic actions assignments $\hat{V}$ and pose assignments $\hat{Z}$ using:
%the following at each step:
%%{\small
%\begin{equation}
% \hat V, \hat Z | y= \argmax_{V,Z} E(X, Z, V, y)
%\end{equation}
%%}
%Therefore, for each possible activity class $y$, we must find $\hat V$ and
%$\hat Z$ using:
%{\small
\begin{equation}
\begin{split}
 \hat{V}, \hat{Z} | y ~ =~ &   \argmax_{V,Z} ~   \sum_{t=1}^T \left( \alpha_{y,v_{t}} 
                  + \beta_{v_{t},z_{t}} + {w_{z_{t}}}^\top x_{t} \delta(z_t \le K)  \right. \\ 
				& \quad\quad \left. \vphantom{{w_{z_{t}}}^\top x_{t}} + \theta \delta(z_t = K+1) + \gamma_{v_{{t-1}},v_t} + \eta_{z_{{t-1}},z_t}  \right). \\
\end{split}
\label{eq:classify_inference}
\end{equation}
%}

% aybe the video representation must be located in experiments
%input{videoRepresentation}
