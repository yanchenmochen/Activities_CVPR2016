\begin{figure}[tb]
\begin{center}
\fbox{\rule{0pt}{2in} \rule{0.9\linewidth}{0pt}}
%\includegraphics[width=0.99\linewidth]{./fig_graphical_hierarchic_model.pdf}
\end{center}
\caption{Graphical representation of our discriminative hierarchical model for 
recognition of composable human activities.
At the top level, activities are represented as compositions of atomic actions that are inferred at
the intermediate level. These actions are in turn compositions of poses at the
lower level, where pose dictionaries are learned from data. Our model also learn
temporal transitions between consecutive poses and actions. Best viewed in
color.}
\label{fig:overview}

\end{figure}


In this section, we introduce our model for pose-based recognition of complex 
human actions. Our goal is to provide the model with the capability of 
annotating input videos according to the actions being performed. In 
particular, we are interested in automatically identifying the parts of the body 
that are involved in each action (spatial localization), as well as the temporal 
span of each action (temporal localization). Since we are interested in 
concurrent and composable activities, we would also like to encode multiple 
levels of abstraction, so that we can encode poses, actions, and their 
compositions. Therefore, we develop a hierarchical compositional framework for 
modeling and recognizing complex human actions.

One of the key contributions of our model is its capability of spatially 
localizing the body regions that are involved in the execution of each action, 
\emph{both at training and testing time}. This is, our training process does not 
require careful spatial annotation and localization of actions in the training 
set. Instead, it uses temporal annotations of actions so 
that, at test time, it can discover the spatial and temporal span, as well as, 
the specific configuration of the main body regions executing each action. In 
the following, we introduce the components of our model and the training 
process that achieves this goal.

\subsection{Body regions}
We divide the body pose into $R$ fixed spatial regions and independently compute 
a pose feature vector for each region. Fig. \ref{fig:skeleton_limbs_regions} 
illustrates the case when $R = 4$ that we use in all our experiments. Our body 
pose feature vector consists of the concatenation of two descriptors. At frame 
$t$ and region $r$, a descriptor $x^{g}_{t,r}$ encodes geometric information 
about the spatial configuration of body joints, and a descriptor $x^{m}_{t,r}$ 
encodes local motion information around each body joint position. Following 
\cite{Lillo2014}, our geometric descriptor is based on angles between segments 
connecting two joints, and angles between these segments and a plane formed by 
three joints (see \cite{Lillo2014} for details). Following \cite{WangCVPR2011}, 
our motion descriptor is based on tracking motion trajectories of key points, in 
our case, joint positions. Specifically, we compute at each joint location a HOF 
using RGB patches centered at the joint location for a temporal window of 15 
frames. At each joint location, this produces a 108-dimensional descriptor,  
that we concatenate across all joints to obtain our motion descriptor. Finally, 
to reduce dimensionality, we apply PCA to transform the concatenated descriptor 
into a 20-dimensional vector, keeping the dimensionality of our final descriptor 
relatively low.


\subsection{Hierarchical compositional model}

We propose a compositional hierarchical model that spans three semantic levels. 
At the top level, our model assumes that each complex action is composed of a 
temporal and spatial arrangement of a known set of $A$ atomic actions, where 
each input video has a single complex action label. Similarly, at the 
intermediate level, our model assumes that each atomic action is composed of a 
temporal and spatial arrangement of $K$ body poses, where $K$ is a parameter of 
our model. Finally, at the bottom level, our model identifies local body poses 
using a bank of linear classifiers that are applied to the incoming frame 
descriptors. 

Similarly to previous works \cite{Lillo2014, Taralova:EtAl:2014}, we build 
each layer of our hierarchical model on top of BoWs 
representations. To this end, at the bottom level of our hierarchy, and for 
each body region, we learn a dictionary of body part configurations that we 
refer to as motion poselets. Similarly, at the mid-level of our hierarchy, and 
for each atomic action, we learn a dictionary of representative action 
configurations that we refer to as actionlets. At each of these levels, 
spatio-temporal activations of the corresponding dictionary words are then used 
to obtain the histograms of the BoWs representations. Next two sections provide 
details behind the process to represent and learn the dictionaries of motion 
poselets and actionlets. Here we provide further details of the 
integrated hiearchical model.

We expresses the resulting hierarchical model using an energy formulation. 
Specifically, given a video $D$ with $T$ frames, we
define an energy function for $D$ as:

\begin{align}\label{Eq_energy}
%\begin{split}
E(D) = & E_{\text{motion poselets}} + E_{\text{motion poselets BoW}} + 
E_{\text{actionlets BoW}} \nonumber \\ 
& + E_{\text{motion poselets transition}} + E_{\text{actionlets 
transition}}.
%\end{split}
\end{align}

In Equation (\ref{Eq_energy}), We 
also 
consider two additional energy potentials that encode information related to 
temporal 
transitions between pairs of motion poselets ($E_{\text{motion poselets 
transition}}$) and 
actionlets ($E_{\text{actionlets transition}}$). Our goal is to find the 
spatial and temporal arrangement 
of motion poselets and actionlets, as well as, the underlying 
complex action, that maximize $E(D)$. 

Using BoW representations and linear classifiers to identify motion poselets, 
the previous energy function is given by:
{\small
\begin{align}
E_{\text{motion poselets}}  =  \sum_{r,t} & \left[ \sum_{k=1}^K {w^r_k}^\top 
x_{t,r}\delta(z_{t,r} = k) + 
\theta^r \delta(z_{t,r}=K+1)\right] \\
E_{\text{motion poselets BoW}} & = \sum_{r=1}^R \sum_{a=1}^A {\beta^r_{a}}^\top 
h^{a}(Z_r,V_r) \\
E_{\text{actionlets BoW}} &=\sum_{r=1}^R {\alpha^r_{y}}^\top h(V_r) \\
E_{\text{motion poselets transition}} & = \sum_{r,k,k'} \eta^r_{k,k'} 
\sum_{t=1}^{T-1} 
\delta(z_{t-1,r}=k)\delta(z_{t,r}=k')\\
E_{\text{actionlets transition}} & =\sum_{r, a,a'} \gamma^r_{a,a'} 
\sum_{t=1}^{T-1} 
\delta(v_{t,r}=a)\delta(v_{t+1,r}=a') 
\end{align}
}


\todo[inline]{IVAN ACA HAY QUE EXPLICAR LA NOTACION DE LAS ECUACIONES 
ANTERIORES}


\subsection{Learning motion poselets}
\todo[inline]{IVAN ACA HAY QUE EXPLICAR QUE LA ECUACION ANTERIOR A NIVEL DE 
POSES ES PROVISTA DE VARIABLES LATENTES}

\subsection{Learning actionlets}
A single linear classifier does not offer enough flexibility to identify 
actions that feature high variability. Consequently, we augment the previous 
hierarchical model to include multiple linear classifiers per action. We create 
two new concepts: \textbf{semantic actions}, that refer to actions \emph{names} 
that compose an activity; and \textbf{atomic sequences}, that refers to the 
sequence of poses that conform an action. Several atomic sequences can be 
associated to a single semantic action, creating disjoint sets of atomic 
sequences, each set associated to a single semantic action.  The main idea is 
that the action annotations in the datasets are associated to semantic actions, 
whereas for each semantic action we learn several atomic sequence classifiers. 
With this formulation, we can handle the multimodal nature of semantic actions, 
covering the changes in motion, poses , or even changes in meaning of the action 
according to the context (e.g. the semantic action ``open'' can be associated to 
opening a can, opening a door, etc.). 

Inspired by \cite{Raptis2012}, we first use the \emph{Cattell's Scree test} for 
finding a suitable number of atomic sequence for every semantic action. Using 
the semantic action labels, we compute a descriptor for every interval using 
normalized histograms of pose labels. Then, for a particular semantic action 
$u$, we compute the the eigenvalues $\lambda_u$ of the affinity matrix of the 
semantic action descriptors, using $\chi^2$ distance. For each semantic action 
$u \in \{1,\dots,U\}$ we find the number of atomic sequences $G_u$ as $G_u = 
\argmin_i \lambda_{i+1}^2 / (\sum_{j=1}^i \lambda_j) + c\cdot i$, with $c=2\cdot 
10^{-3}$. Finally, we cluster the descriptors corresponding to each semantic 
action using k-means, using a different number of clusters for each semantic 
action $u$ according to $G_u$. This approach generates non-overlapping atomic 
sequences, each associated to a single semantic action.

To transfer the new labels to the model, we define $u(v)$ as the function that 
given the atomic sequence label $v$, returns the corresponding semantic action 
label $u$. The energy for the activity level is then
\begin{equation}
E_{\text{activity}} =  \sum_{u=1}^U\sum_{t=1}^T \alpha_{y,u}\delta(u(v_t)=u)
\end{equation}  

For the action and pose labels the model remains unchanged. Using the new atomic 
sequences allows a richer representation for actions, while in he activity 
level, several atomic sequences will map to a single semantic action. This 
behavior resembles a max-pooling operation, where we will choose at inference 
the atomic sequences that best describe the performed actions in the video, 
keeping the semantics of the original labels. 

\subsection{Garbage collector}

The model in \cite{Lillo2014} uses all poses to feed action classifiers. Out 
intuition is that only a subset of poses in each video are really discriminative 
or informative for the actions performed, while there is plenty of poses that 
corresponds to noisy or non-informative ones. [EXPAND] Our intuition is that 
low-scored frames in terms of poses (i.e. a low value of $w_{z_t}^\top x_t$ in 
Eq. (\ref{eq:energy2014})) make the same contribution as high-scored poses in 
higher levels of the model, while degrading the pose classifiers at the same 
time since low-scored poses are likely to be related to non-informative frames. 
We propose to include a new pose, to explicitly handling those low-scored 
frames, keeping them apart for the pose classifiers $w$, but still adding a 
fixed score to the energy function to avoid normalization issues and to help in 
the specialization of pose classifiers. We call this change in the model a 
\emph{garbage collector} since it handles all low-scores frames and group them 
having a fixed energy score $\theta$. In practice, we use a special pose entry 
$K+1$ to identify the non-informative poses. The equation representing the 
energy for pose level is
%
\begin{equation} \label{Eq_poseEnergy}
E_{\text{poses}}  = \sum_{t=1}^T \left[  {w_{z_t}}^\top x_{t}\delta(z_{t} \le  
K) + \theta 
\delta(z_{t}=K+1)\right] 
\end{equation}
where $\delta(\ell) = 1$ if $\ell$ is true and $\delta(\ell) = 0$ if
$\ell$ is false. The action level also change its energy:
\begin{equation}
\begin{split}
 \label{Eq_actionEnergy}
E_{\text{actions}} =  \sum_{t=1}^T \sum_{a=1}^A \sum_{k=1}^{K+1}  \beta_{a,k} 
\delta(z_t = k) \delta(v_t = a).
\end{split}
\end{equation}



\subsection{Learning} \label{subsec:learning}

[IL: These are notes, don't evaluate it yet]

The loss function $\Delta((y_i,V_i),(y,V))$ can no longer depend in the value of $V_i$, since it is now a latent variable. But, as we do know the time span of the actions in all videos, we can compute a list $A_t$ of possible actions for frame $t$, and transform the original loss function into
\begin{equation}
\Delta(y_i,(y,V)) = \lambda_y(y_i \ne y) + \lambda_v\frac{1}{T}\sum_{t=1}^T \delta(v_t \notin A_t)
\end{equation}
and use the same group of actions for each frame to impute the labels of actions and poses for each frame during the inference of latent variables

The model parameters are learned with a Latent Structural SVM formulation, iterating between searching the best assignments of poses $Z$ given the model parameters, and finding the best classifiers given the poses $Z$:
\begin{equation}
 Z_i^* = \max_{Z_i} E(X_i, Z_i, V_i, y_i)
\end{equation} 

\subsection{Inference}
The input to the inference algorithm is a new video sequence with features
$X$. The task is to infer the best activity label $\hat y$ and the best
atomic action labels $\hat V$. Additionally, we also need to estimate latent variables $Z$.
%{\small
\begin{equation}
  \hat y, \hat V, \hat Z = \argmax_{y, V, Z} E(X, Z, V, y)
\end{equation}
%}
We can solve this by the same equations for solving the most violated constraint during learning, setting %as in Equation (\ref{dp_recursion}), using 
$\lambda_i =0$, $i = \{1,2\}$, by exhaustively enumerating all values of activities $y$, and solving for atomic actions assignments $\hat{V}$ and pose assignments $\hat{Z}$ using:
%the following at each step:
%%{\small
%\begin{equation}
% \hat V, \hat Z | y= \argmax_{V,Z} E(X, Z, V, y)
%\end{equation}
%%}
%Therefore, for each possible activity class $y$, we must find $\hat V$ and
%$\hat Z$ using:
%{\small
\begin{equation}
\begin{split}
 \hat{V}, \hat{Z} | y ~ =~ &   \argmax_{V,Z} ~   \sum_{t=1}^T \left( \alpha_{y,v_{t}} 
                  + \beta_{v_{t},z_{t}} + {w_{z_{t}}}^\top x_{t} \delta(z_t \le K)  \right. \\ 
				& \quad\quad \left. \vphantom{{w_{z_{t}}}^\top x_{t}} + \theta \delta(z_t = K+1) + \gamma_{v_{{t-1}},v_t} + \eta_{z_{{t-1}},z_t}  \right). \\
\end{split}
\label{eq:classify_inference}
\end{equation}
%} 






